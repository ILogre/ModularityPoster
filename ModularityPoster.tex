\documentclass{acm_proc_article-sp}
\usepackage{seb}
\usepackage{fancyref}

\begin{document}


\title{Composition Challenges for Sensor Data Visualization}
%\subtitle{}

\numberofauthors{4}
\author{
% 1st. author
\alignauthor Ivan Logre\\
       \affaddr{Univ. Nice Sophia Antipolis}\\
       \affaddr{CNRS, I3S, UMR 7271}\\
       \affaddr{06900 Sophia Antipolis}\\
       \email{logre@i3s.unice.fr}
% 2nd. author
\alignauthor S{\'e}bastien Mosser\\
       \affaddr{Univ. Nice Sophia Antipolis}\\
       \affaddr{CNRS, I3S, UMR 7271}\\
       \affaddr{06900 Sophia Antipolis}\\
       \email{mosser@i3s.unice.fr}
% 3rd. author
\alignauthor Michel Riveill\\
       \affaddr{Univ. Nice Sophia Antipolis}\\
       \affaddr{CNRS, I3S, UMR 7271}\\
       \affaddr{06900 Sophia Antipolis}\\
       \email{riveill@i3s.unice.fr}
}

\date{06 January 2015}

\maketitle
\begin{abstract}
This paper describes how the design of dashboard visualizing data collected from
sensors triggers multiple challenges about software composition through separation
of concern and modularity in order to insure properties on the design process.
\end{abstract}

% A category with the (minimum) three required fields
\category{D.2.8}{Software Engineering}{Data visualization, Software composition}[]

\keywords{visualization, sensor, data, composition}

\section{Introduction}
The increase of data producing objects leads to a growing need to interpret those
data in order to extract knowledge. When there is a human at the end of the
production chain, the result must allow her to exploit it, so being adapted to his
capabilities. Such visualization are used to assist one to make a decision according
to a projection of facts, as are sensed data. Visualization is a way to allow a
human to induce information from raw data through dashboards. The first chapter
defines our context of work through a short definition of this field, a clarification
about the aim and the efforts to achieve it and several exiting way to implement
a visualization solution. The second chapter raises the challenges relatively to
software engineering, detailing properties to observe while contributing to this
particular field. Then, the third chapter briefly offers some perspectives to
tackle those challenges.

\section{Sensor data visualization}


\subsection{What is sensor data visualization ?}
% Data visualization through dashboards (def)

Stephen Few define a dashboard as "a visual display of the most important
information needed to achieve one or more objectives; consolidated and arranged
on a single screen so the information can be monitored at a glance"\cite{few:dashboard}
In this paper, we will discuss about multiple dashboards, i.e. succession of screens,
in order to include in our study the sensor data rendering organised in separate unit
of visualization.

% What kind of visualizations ?
According to this definition, the visualizations used for this purpose should
be a graphical representation of the data to treat. Apart from the well-known
generic charts (e.g. line chart, bar chart, pie chart, ...), a huge amount
of specialized visualizations is available for specific data and/or task.
% Static vs dynamic
Such graphical widgets may present a static representation of the data, as an
assessment of a situation, or may offer one to interact with it with an aim of
exploration, to search a more relevant data for each use, or to offer both an
overview and a way to reduce the scope of the visualization.
% Nb of dimensions
Besides the ornamental criteria which only consist on a effect to change the
aspect of the widget without impacting it capabilities (eg. 3D pie chart), the
number of dimensions displayed play a part in the final use of the visualizations.
Indeed, there may be one dimension (e.g. a simple display of your current speed
on your car dashboard), two dimensions (as the generic charts above), or more
(e.g. a country map with a color gradient to weight the population). 
Among the dimensions of a given visualization, one or several has to be used to
categorize the input data. The other dimensions are the values to display.
For example, a line chart represents values according to the time where
a pie chart display, for each entry of a given categorization, the volume
or percentage of matching data. The previously mentioned three dimension map
displays a value, the amount of human being, in function of a complex category :
the combination of a latitude and a longitude.
In addition, due to ergonomic constraints, a visualization may impose a limit on
(i) the number of different category displayable, e.g. a pie chart should
have eight portions maximum;
(ii) the number of value dimensions, e.g. a line chart should not represent more
than ten lines to be readable
(iii) on both category and value number, e.g. a bar chart should displays maximum
five values per category, and the number of category should be limited according
to the screen size so that the user can still distinguish each bar from the others.

% For what kind of data ?
Raw data produce by physical sensors are essentially timed values. The type of
the value (eg. string, numerical, ...) and the unit of measurement in which it
is expressed have an influence on the suitable possibilities of visualization.
For example, given a window status sensor emitting either "opened" or "close"
every minute, the use of a pie chart would not be relevant. On the other hand,
knowing that the values are temperature measure may lead to a continuous representation
(e.g. line chart) instead of a discrete one (e.g. a bar chart), or even to a specialized
widget for this kind of data.
% Virtual sensor & Data composition
Data can also be the result of a calculation, which can be as simple as the transformation
of an electrical tension to a temperature, or a more complex one such as a composition
of several dataset to infer a new one.

% For what purposes ?
The intention pursued in the dashboard construction is directly linked to design
choices and to the selection visualization among the amount of possibilities available.
Indeed, two different widgets, fed with the exact same data, may highlight specific
aspect of it and lead to different domains of interpretation. For example,
the use of a line chart implicitly infer the impression of continuity and so leads
to focus the interpretation on the variations in the values while, in a other hand,
using a bar chart will result in an analysis of the comparison between several values
for the same category.

\subsection{How to achieve it ?}
In order to design such visualization dashboard, one need to chose visualizations
to apply to refined datasets in order to achieve some identified final goal.
This lead to several tasks, independently of the concrete methodology one chooses to use :
\iitem{a} define what will be the purpose of the dashboard
\iitem{b} select data to be visualize
\iitem{c} chose visualization for each group of data
\iitem{d} fill the visualization with the well formatted data set
\iitem{e} arrange the visualisations into a (multiple) dashboard
\iitem{f} audit the dashboard according to the initial motivation

\subsection{How to implement it}
Those tasks refer to three different domains :\\
Requirement Engineering \iitem{a} \& \iitem{f}\\
Data treatment \iitem{b} \& \iitem{d}\\
Dashboard design \iitem{c} \& \iitem{e}\\
The separation of the whole problem in domain specific sub-part aim to simplify
the resolution of each domain challenge and to allow the contribution of specialist
for a domain specific problem without disturbance from the other part noise.
% existing solution
One could use existing solutions to implement each of those tasks, for example
using SQL would be a suitable choice to store and refine sensor data, IFML to
model the wanted organization of the dashboard and temporal logic to define
the succession of action aimed to be performed with the final product, and finally
D3.JS, AmChart or an other visualization library to implement the widgets
with some HTML5 code to structure the result.
% DSML
One other way to tackle it is to design one DSL per domain
This way, one, when in charge of a given domain, may deal with a language
specifically designed for his task, without being polluted by others
vocabulary or domain constraints.

\section{Properties to satisfy and associated challenges}
However, the design of such visualization tool is not trivial, leading to several challenges.

\subsection{Evolutivity}
Data visualization is a growing, fast evolving field, as well as requirement
engineering is an active research field offering new way to capture needs.
For those reasons, their respective DSL will have to evolve in time, the
composition should therefore take this fact into consideration to ease the
evolutivity of the whole composed dashboard design framework.

\subsection{Isolation}
Each domain expert (i.e. user of a solution specific to a domain) should be able to
work in isolation from the other domains. This means that one can always contribute
in her own domain, to improve her contribution, and being able to check that
this evolution is consistent in this domain, even if the other domains are in
an unstable or incoherent state.

\subsection{Low coupling}
The number of hypothesis made in a domain implementation to the others should be
as small as possible. In addition to be a good software development practice,
this criteria eases the replacement of solution by an other one, e.g. of the
data management by a visualization resource DSL.
									%%% to complete

\subsection{Integrity}

\section{Actionable insights}
\subsection{DSL composition}
One interesting way to tackle those challenges would be to design a DSL for each
of the three domains mentioned in \ref{subsec:How to implement it} and then to
compose those partial results in a overall data visualization solution.
The state of the art reveals two main ways to manage this composition :\\
\iitem{i} Fusion/Merge, i.e. the operation to produce one bigger meta model
from several meta models by identification of a pivot and merging from it, and
then refine the associated concrete syntaxes to produce a global one. (insert ref)\\
\iitem{ii} Aggregation, i.e. make enough assumptions about the meta models to be able to link them through the transformation of several meta models by adding, deleting or editing specific model elements, essentially to align two concepts from different domains or to reference an external concept in order to delegate part of the responsibilities. (insert ref)\\
The authors short-term perspective is to work on the integration of the DSLs as software services in order to be able to compose languages while insuring the same properties as service-oriented architecture field.



% The following two commands are all you need in the
% initial runs of your .tex file to
% produce the bibliography for the citations in your paper.
\bibliographystyle{abbrv}
\bibliography{sigproc}  % sigproc.bib is the name of the Bibliography in this case
% You must have a proper ".bib" file
%  and remember to run:
% latex bibtex latex latex
% to resolve all references
%
% ACM needs 'a single self-contained file'!
\end{document}
