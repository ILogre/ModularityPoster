\documentclass{sigplanconf}
%\documentclass{acm_proc_article-sp}
\usepackage{seb}
\usepackage{fancyref}
\usepackage{mdwlist}
\makecompactlist{myenumerate}{enumerate}
\usepackage{url}
\begin{document}

\special{papersize=8.5in,11in}
\setlength{\pdfpageheight}{\paperheight}
\setlength{\pdfpagewidth}{\paperwidth}

\conferenceinfo{MODULARITY '15}{March 16--19, 2015, Fort Collins, Colorado, USA} 
\copyrightyear{2015}
\copyrightdata{978-1-nnnn-nnnn-n/yy/mm} 
\doi{nnnnnnn.nnnnnnn}
\exclusivelicense 

\title{Composition Challenges for Sensor Data Visualization}

\authorinfo{Ivan Logre \and S{\'e}bastien Mosser \and Michel Riveill}
  		   {Univ. Nice Sophia Antipolis\\CNRS, I3S, UMR 7271\\06900 Sophia Antipolis}
  		   {\emph{lastname}@i3s.unice.fr}
\maketitle

\begin{abstract}
  %% Problem
  Connected objects and monitoring systems continuously produce
  data about their environment. Dashboards are then designed to
  aggregate and present these data to end-users.
  %% Contribution
  Technologies used to design and implement visualization dashboards
  are babbling from a software engineering point of view. This paper
  highlights how this domain could benefit from leveraging separation
  of concerns and software composition paradigms to support dashboard design.
  %% Results
  % It describes how the design of dashboard visualizing data
  % collected from sensors triggers multiple challenges about software
  % composition through separation of concern and modularity in order to
  % insure properties on the design process.
\end{abstract}

% A category with the (minimum) three required fields
\category{D.2.8}{Software Engineering}{Data visualization, Software
  composition}[]

\keywords{visualization, sensor, data, composition}

\section{Introduction}

The \emph{Internet of Things} relies on physical objects interconnected
between each others, creating a mesh of devices producing
information. In this context, sensors are surrounding our environment
(\eg cars, buildings, smartphones) and continuously collect data about
our living environment. In order to add value to these raw data sets,
visualization dashboards are designed to support end-user decision
making process. Unfortunately, the tools available to design and
implement such dashboards are holistic and do not take into account
the inherent modularity of this domain. This paper does not aim to
describe a solution, but instead focuses on the challenges triggered
by the design of visualization dashboards, and align them with
modular paradigms such as separation of concerns and software
composition.

% The increase of data producing objects leads to a growing need to
% interpret those data in order to extract knowledge. When there is a
% human at the end of the production chain, the result must allow her to
% exploit it, so being adapted to his capabilities. Such visualization
% are used to assist one to make a decision according to a projection of
% facts, as are sensed data. Visualization is a way to allow a human to
% induce information from raw data through dashboards. The first chapter
% defines our context of work through a short definition of this field,
% a clarification about the aim and the efforts to achieve it and
% several exiting way to implement a visualization solution. The second
% chapter raises the challenges relatively to software engineering,
% detailing properties to observe while contributing to this particular
% field. Then, the third chapter briefly offers some perspectives to
% tackle those challenges.

\section{Visualization Dashboards}
% une colonne max.
%\mypar{Process overview}
\textbf{Process overview.}
% Qui sont les roles intervenants?
To design and implement a visualization dashboard, one selects the
visualizations applied to refined datasets to achieve some identified
goal. 

This process involves three roles: \begin{myenumerate}
\item a \emph{Requirement Engineer} (RE);
\item a \emph{Data Manager} (DM);
\item a \emph{Dashboard Designer} (DD).
\end{myenumerate}
% Quels sont leurs domaines de competences?
Each role is responsible for several tasks: \emph{(i)} the RE defines what
will be the purpose of the dashboard and audit the resulting dashboard
according to the initial motivation, \emph{(ii)} the DM selects data to be
visualized and treat it to provide the needed well formatted
datasets, \emph{(iii)} the DD choose visualization for each group of data and
arrange them spatially into a dashboard.
% Pourquoi ces competences sont souvent des acteurs differents et pas
% juste des roles?
These roles needs to collaborate, \eg the choice of visualization the
DD has to make depends on the purpose exposed by the RE. They rely
on distinct domains of expertise and formation, each one bringing its
own challenges.Thus, each role is usually impersonated by a
dedicated stakeholder.  

\textbf{Designing dashboard.}
% \begin{enumerate}
%   Quels sont les outils a dispositions (UML, CTT)
One could use existing solutions to implement each of the mentioned
tasks, for example using SQL would be a suitable choice to query and
refine sensor data. The \emph{Interaction Flow Modeling
  Language}~\cite{ifml} models the wanted organization of the
dashboard, and temporal logic or \emph{Concurrent Tasks Tree}
(CTT)~\cite{ctt} define the aimed product and the sequence of actions
to be performed with it.
% Quels sont les problemes quand on les utilise?
Those tools were not designed for the visualization field and suffers
from their generic approaches by being difficult to use by
visualization designers. For example CTT does not offer concepts to
characterize visualization needs.
% Est-ce qu'on sort certains experts de leur domaine de competences
% avec ces outils?
Moreover, there is a lack of interoperability between those tools due
to the distinct field they come from. This results in a difficulty for
the stakeholders to dialogue and converge toward a solution when a
compromise is needed.
% \end{enumerate}

\textbf{Implementing dashboard.}
% Quels sont les outils a dispositions (Web, bibliotheques)
To implement a given dashboard, one can use visualization widget
libraries, either professional solutions such as
HighChart\footnote{\url{http://www.highcharts.com/}} and
AmChart\footnote{\url{http://www.amcharts.com/}} or com\-munity-based
libraries such as D3.JS\footnote{\url{http://d3js.org/}}. Then, one
will add HTML5/CSS code to structure the result.
% Quels sont les problemes quand on les utilise?
However, those widgets do not allow their integration with a lot of
data format, since the development effort is put on the interaction
aspect instead of the interoperability. In addition, the huge amount
of available widgets (\eg D3.js offers 235 widgets on January 2015)
increase the difficulty to select a suitable visualization. There is 
%Est-ce qu'on sort certains experts de leur domaine de competences avec ces outils?
a lack of effort in the categorization of those new visualization
capabilities\cite{ecmfa}.  These last two points strengthen the
difficulty to cooperate with other domains, considering the gap
between the conceptual role of the RE and the implementation role of
the DD, and because of the incompatible constraints imposed by the chosen
libraries on data format then reduce reusability.

\section{Challenges}

The previous section described how people support the design and
implementation of dashboards using a classical development process. In
this section, we describe the challenge of isolation that undermine
this process, detailing it through two axes: evolution capacities and
dashboard integrity. We discuss how separation of concerns and
software composition could be helpful to support this process.

%% Pour chaque paragraphe, se raccrocher au contexte de la section
%% d'avant (je ne peux pas le faire tant que c'est aps ecrit),
%% identifier le probleme du point de vue ou de la separation ou de la
%% composition (ou des deux), et discuter des benefics envisages par
%% une approche de ce genre.

\textbf{Isolation challenge.}  Each stakeholder should be able to work
in his domain of expertise, isolated from the other domains. This main
challenge implies that each one work without the irrelevant noise of
others' contributions, focusing on concepts of her own domain, or, in
case of a shared concept, only on the facet of this concept relevant
for the task of this role.  For example, as a dashboard designer, one
can define a link between the visualization being constructed and some
data, but should not be requested to be competent as a data manager
while working on it.  The relevant information from a data polishing
point of view are handled by a specialist of this domain, but not only
this specialist would need to reference a specific dataset.
% quels sont les avantages a les maintenir séparés ^

\textbf{Evolution capabilities.}  Data visualization is a growing,
fast evolving field: 6,440,000 Google results for \emph{"data
  visualization"} on 01/15, D3.js offered 133 widgets on 04/14 and now
235 on 01/15. \emph{Requirement engineering} (RE) is a very active
research field offering new way to capture needs: more than 730 papers
published since 2014 and referenced by Google Scholar contains
\emph{"requirement engineering"}, and 640 \emph{"goal model"}. In
addition, data scalability is still an open scientific lock. For these
reasons, their respective tool or \emph{Domain Specific Languages}
(DSL) used have to evolve in time.  Exploiting a composition-based
approach to represent the widgets will support the evolution
capabilities of the whole design framework. In addition, each of these
domains bring a unique expertise useful to the data visualization
design field. Keeping them separated allows each stakeholder to
perform dedicated tasks with state of the art capabilities of the
associated domain.  Nonetheless, this separation of role specific
solutions require to handle the interaction between those partial
results, bridging the gap between the domains and manage the high
versatility of those research fields.

\textbf{Integrity.} Separation of concerns allows one to contribute to
the dashboard design process in her own domain. It is then possible to
check if this work is consistent inside this domain, even if the other
domains are in an unstable or incoherent state or if it is not
possible to check the global consistency at the moment. For example, a
data manager is allowed to edit a resource measurement unit to
optimize the data transfer and validate her contribution from the data
point of view, even if it may have broke the choice of visualizations
used in the dashboard design domain.  In order to support a proper
separation of concerns, each domain solution has then to be usable
independently from any considerations of interaction with the other
collaborating domains.  This point raises a challenge on the
interaction of these partial solutions, introducing the notion of
local and global consistency to handle.

%% Ouvrir sur la compo de DSL
\textbf{Actionable insights: DSL composition.}  One interesting way to
tackle those challenges would be to design a DSL for each of the three
domains mentioned and then to compose those partial results in a
overall data visualization solution.
The state of the art reveals two main ways to manage this composition:\\
\iitem{i} Merge, \ie the operation to produce one bigger meta model
from several meta models by identification of a pivot and merging from
it, and then refine the associated concrete syntaxes to produce a global 
one~\cite{kienzle13}.\\
\iitem{ii} Aggregation, \ie make enough assumptions about the
meta-models to be able to link them through the transformation of
several meta-models by adding, deleting or editing specific model
elements, essentially to align two concepts from different domains or
to reference an external concept
in order to delegate part of the responsibilities~\cite{blouin14}.

Our short-term perspective is to work on the integration of the DSLs
as software services in order to be able to compose languages while
insuring the same properties as service-oriented architecture field.

\section{Conclusions \& Perspectives}
In this paper, we described the domain of visualization dashboard
design. This domain crosscuts several research fields, from
human-computer interactions to big data. We highlighted three
challenges in this domain where a software engineering approach based
on modularity concepts could support it. However, all the challenges
triggered by this domain are not yet solved from a separation of
concerns point of view. In our upcoming works, we plan to focus on
defining a formal way to support exchanges between the different roles
involved in the domain, emphasizing integrity and isolation
properties.
%% Un petit paragraphe de conclusions/



% The following two commands are all you need in the initial runs of
% your .tex file to produce the bibliography for the citations in your
% paper.
\bibliographystyle{abbrv}
\bibliography{sigproc} % sigproc.bib is the name of the Bibliography in this case
% You must have a proper ".bib" file and remember to run: latex bibtex
% latex latex to resolve all references
%
% ACM needs 'a single self-contained file'!
\end{document}
