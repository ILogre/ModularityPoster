% \begin{meta-comment}
%
% $Id: mdwlist.dtx,v 1.1 1996/11/19 20:52:26 mdw Exp $
%
% Various list-related things
%
% (c) 1996 Mark Wooding
%
%----- Revision history -----------------------------------------------------
%
% $Log: mdwlist.dtx,v $
% Revision 1.1  1996/11/19 20:52:26  mdw
% Initial revision
%
%
% \end{meta-comment}
%
% \begin{meta-comment} <general public licence>
%%
%% mdwlist package -- various list-related things
%% Copyright (c) 1996 Mark Wooding
%%
%% This program is free software; you can redistribute it and/or modify
%% it under the terms of the GNU General Public License as published by
%% the Free Software Foundation; either version 2 of the License, or
%% (at your option) any later version.
%%
%% This program is distributed in the hope that it will be useful,
%% but WITHOUT ANY WARRANTY; without even the implied warranty of
%% MERCHANTABILITY or FITNESS FOR A PARTICULAR PURPOSE.  See the
%% GNU General Public License for more details.
%%
%% You should have received a copy of the GNU General Public License
%% along with this program; if not, write to the Free Software
%% Foundation, Inc., 675 Mass Ave, Cambridge, MA 02139, USA.
%%
% \end{meta-comment}
%
% \begin{meta-comment} <Package preambles>
%<+package>\NeedsTeXFormat{LaTeX2e}
%<+package>\ProvidesPackage{mdwlist}
%<+package>                [1996/05/02 1.1 Various list-related things]
% \end{meta-comment}
%
% \CheckSum{179}
%% \CharacterTable
%%  {Upper-case    \A\B\C\D\E\F\G\H\I\J\K\L\M\N\O\P\Q\R\S\T\U\V\W\X\Y\Z
%%   Lower-case    \a\b\c\d\e\f\g\h\i\j\k\l\m\n\o\p\q\r\s\t\u\v\w\x\y\z
%%   Digits        \0\1\2\3\4\5\6\7\8\9
%%   Exclamation   \!     Double quote  \"     Hash (number) \#
%%   Dollar        \$     Percent       \%     Ampersand     \&
%%   Acute accent  \'     Left paren    \(     Right paren   \)
%%   Asterisk      \*     Plus          \+     Comma         \,
%%   Minus         \-     Point         \.     Solidus       \/
%%   Colon         \:     Semicolon     \;     Less than     \<
%%   Equals        \=     Greater than  \>     Question mark \?
%%   Commercial at \@     Left bracket  \[     Backslash     \\
%%   Right bracket \]     Circumflex    \^     Underscore    \_
%%   Grave accent  \`     Left brace    \{     Vertical bar  \|
%%   Right brace   \}     Tilde         \~}
%%
%
% \begin{meta-comment}
%
%<*driver>
\input{mdwtools}
\describespackage{mdwlist}
\def\defaultdesc{%
  \desclabelwidth{80pt}%
  \desclabelstyle\nextlinelabel%
  \def\makelabel{\bfseries}%
}
\newenvironment{cmdlist}
  {\basedescript{\let\makelabel\cmd}}
  {\endbasedescript}
\mdwdoc
%</driver>
%
% \end{meta-comment}
%
% \section{User guide}
%
% This package provides some vaguely useful list-related commands and
% environments:
% \begin{itemize*}
% \item A way of building \env{description}-like environments.
% \item Commands for making `compacted' versions of list environments
% \item A method for suspending and resuming enumerated lists.
% \end{itemize*}
%
% \subsection{Description list handling}
%
% Different sorts of description-type lists require different sorts of
% formatting: I think that's fairly obvious.  There are essentially three
% different attributes which should be changable:
% \begin{itemize*}
% \item the indentation of the items being described,
% \item the handling of labels which don't fit properly, and
% \item the style used to typeset the label text.
% \end{itemize*}
% The first two items should usually be decided for all description-like
% lists in the document, to ensure consistency of appearance.  The last
% depends much more on the content of the labels.
%
% \DescribeEnv{basedescript}
% The \env{basedescript} environment acts as a `skeleton' for description
% environments.  It takes one argument, which contains declarations to
% be performed while constructing the list.  I'd consider it unusual for
% the \env{basedescript} environment to be used in the main text: it's
% intended to be used to build other environments.
%
% The declarations which can be used to define description-type environments
% include all of those which are allowed when setting up a list (see the
% \LaTeX\ book for information here).  Some others, which apply specifically
% to description lists, are also provided:
%
% \begin{itemize}
%
% \item \DescribeMacro{\desclabelwidth}
%       The \syntax{"\\desclabelwidth{"<length>"}"} declaration sets labels
%       to be left-aligned, with a standard width of \<length>; the item
%       text is indented by \<length> plus the value of |\labelsep|.
%
% \item \DescribeMacro{\desclabelstyle}
%       The label style determines how overlong labels are typeset.  A style
%       may be set using the \syntax{"\\desclabelstyle{"<style>"}"}
%       declaration.  The following \<style>s are provided:
%       \begin{cmdlist}
%       \item [\nextlinelabel] If the label is too wide to fit next to the
%             first line of text, then it is placed on a line by itself;
%             the main text is started on the next line with the usual
%             indentation.
%       \item [\multilinelabel] The label is typeset in a parbox with the
%             appropriate width; if it won't fit on one line, then the
%             text will be split onto subsequent lines.
%       \item [\pushlabel] If the label is too wide to fit in the space
%             allocated to it, the start of the item's text will be `pushed'
%             over to the right to provide space for the label.  This is
%             the standard \LaTeX\ \env{description} behaviour.
%       \end{cmdlist}
%
% \item \DescribeMacro{\makelabel}
%       The |\makelabel| command is responsible for typesetting a label.
%       It is given one argument, which is the text given as an argument
%       to the |\item| command; it should typeset it appropriately.  The
%       text will then be arranged appropriately according to the chosen
%       label style.  This command should be redefined using |\renewcommand|.
%
% \end{itemize}
%
% \begin{figure}
% \begin{demo}[w]{Various labelling styles}
%\begin{basedescript}{\desclabelstyle{\nextlinelabel}}
%\item [Short label] This is a short item, although it has quite a
%      lot of text attached to it.
%\item [Slightly longer label text] This is a rather longer piece
%      of text, with a correspondingly slightly longer label.
%\end{basedescript}
%\medskip
%\begin{basedescript}{\desclabelstyle{\multilinelabel}}
%\item [Short label] This is a short item, although it has quite a
%      lot of text attached to it.
%\item [Slightly longer label text] This is a rather longer piece
%      of text, with a correspondingly slightly longer label.
%\end{basedescript}
%\medskip
%\begin{basedescript}{\desclabelstyle{\pushlabel}}
%\item [Short label] This is a short item, although it has quite a
%      lot of text attached to it.
%\item [Slightly longer label text] This is a rather longer piece
%      of text, with a correspondingly slightly longer label.
%\end{basedescript}
% \end{demo}
% \end{figure}
%
% \DescribeMacro{\defaultdesc}
% To allow document designers to control the global appearance of description
% lists, the |\defaultdesc| command may be redefined; it is called while
% setting up a new \env{basedescript} list, before performing the user's
% declarations.  By default, it attempts to emulate the standard \LaTeX\
% \env{description} environment:\footnote{^^A
%   This is a slightly sanitised version of the real definition, which is
%   given in the implementation section of this document.}
% \begin{listing}
%\providecommand{\defaultdesc}{%
%  \desclabelstyle{\pushlabel}%
%  \renewcommand{\makelabel}[1]{\bfseries##1}%
%  \setlength{\labelwidth}{0pt}%
%}
% \end{listing}
% Unfortunately, \LaTeX\ doesn't provide a means for overriding a command
% which may or may not have been defined yet; in this case, I'd probably
% recommend using the \TeX\ primitive |\def| to redefine |\defaultdesc|.
%
% If you want to redefine the \env{description} environment in terms of
% the commands in this package, the following method is recommended:
% \begin{listing}
%\renewenvironment{description}{%
%  \begin{basedescript}{%
%    \renewcommand{\makelabel}[1]{\bfseries##1}%
%  }%
%}{%
%  \end{basedescript}%
%}
% \end{listing}
% This ensures that labels are typeset in bold, as is usual, but other
% properties of the list are determined by the overall document style.
%
% \subsection{Compacted lists}
%
% \LaTeX\ tends to leave a certain amount of vertical space between list
% items.  While this is normally correct for lists in which the items are
% several lines long, it tends to look odd if all or almost all the items
% are only one line long.
%
% \DescribeMacro{\makecompactlist}
% The command
% \syntax{"\\makecompactlist{"<new-env-name>"}{"<old-env-name>"}"}
% defines a new environment \<new-env-name> to be a `compacted' version of
% the existing environment \<old-env-name>; i.e., the two environments are
% the same except that the compacted version leaves no space between items
% or paragraphs within the list.
%
% \DescribeEnv{itemize*}
% \DescribeEnv{enumerate*}
% \DescribeEnv{description*}
% So that the most common cases are already handled, the package creates
% compacted $*$-variants of the \env{itemize}, \env{enumerate} and
% \env{description} environments.  These were created using the commands
% \begin{listing}
%\makecompactlist{itemize*}{itemize}
%\makecompactlist{enumerate*}{enumerate}
%\makecompactlist{description*}{description}
% \end{listing}
%
% Some list environments accept arguments.  You can pass an argument to a
% list environment using an optional argument to its compact variant.  For
% example,
% \begin{listing}
%\begin{foolist*}[{someargument}]
% \end{listing}
%
% \subsection{Suspending and resuming list environments}
%
% \DescribeMacro{\suspend}
% \DescribeMacro{\resume}
% The |\suspend| and |\resume| commands allow you to temporarily end a list
% environment and then pick it up where you left off.  The syntax is fairly
% simple:
%
% \begin{grammar}
%
% <suspend-cmd> ::= \[[
%   "\\suspend"
%   \begin{stack} \\ "[" <name> "]" \end{stack} "{" <env-name> "}"
% \]]%
%
% <resume-cmd> ::= \[[
%   "\\resume"
%   \begin{stack} \\ "[" <name> "]" \end{stack} "{" <env-name> "}"
%   \begin{stack} \\ "[" <text> "]" \end{stack}
% \]]%
%
% \end{grammar}
%
% The \<env-name> is the name of the environment; this will more often than
% not be the \env{enumerate} environment.  The \<name> is a magic name you
% can use to identify the suspended environment; if you don't specify this,
% the environment name is used instead.
%
% \begin{demo}{Suspended environments}
%Here's some initial text.  It's
%not very interesting.
%\begin{enumerate*}
%\item This is an item.
%\item This is another.
%\suspend{enumerate*}
%Some more commentry text.
%\resume{enumerate*}
%\item Another item.
%\end{enumerate*}
% \end{demo}
%
% You can pass arguments to a resumed list environment through the second
% optional argument of the |\resume| command.  If, for example, you're using
% David Carlisle's \package{enumerate} package, you could say something like
% \begin{listing}
%\begin{enumerate}[\bfseries{Item} i]
%\item An item
%\item Another item
%\suspend{enumerate}
%Some intervening text.
%\resume{enumerate}[{[\bfseries{Item} i]}]
%\item Yet another item
%\end{enumerate}
% \end{listing}
%
% \implementation
%
% \section{Implementation}
%
%    \begin{macrocode}
%<*package>
%    \end{macrocode}
%
% \subsection{Description lists}
%
% \subsubsection{Label styles}
%
% \begin{macro}{\nextlinelabel}
%
% The idea here is that if the label is too long to fit in its box, we put
% it on its own line and start the text of the item on the next.  I've
% used |\sbox| here to capture colour changes properly, even though I have
% deep moral objections to the use of \LaTeX\ boxing commands.  Anyway,
% I capture the text in box~0 and compare its width to the amount of space
% I have in the label box.  If there's enough, I can just unbox the box;
% otherwise I build a vbox containing the label text and an empty hbox --
% |\baselineskip| glue inserted between the two boxes makes sure we get
% the correct spacing between the two lines, and the vboxness of the vbox
% ensures that the baseline of my strange thing is the baseline of the
% \emph{bottom} box.  I then bash the vbox on the nose, so as to make its
% width zero, and leave that as the result.  Either way, I then add glue
% to left align whatever it is I've created.
%
%    \begin{macrocode}
\def\nextlinelabel#1{%
  \sbox\z@{#1}%
  \ifdim\wd\z@>\labelwidth%
    \setbox\z@\vbox{\box\z@\hbox{}}%
    \wd\z@\z@%
    \box\z@%
  \else%
    \unhbox\z@%
  \fi%
  \hfil%
}
%    \end{macrocode}
%
% \end{macro}
%
% \begin{macro}{\multilinelabel}
%
% A different idea -- make the label text wrap around onto the next line if
% it's too long.  This is really easy, actually.  I use a parbox to contain
% the label text, set to be ragged right, because there won't be enough
% space to do proper justification.  There's also a funny hskip there --
% this is because \TeX\ only hyphenates things it finds sitting \emph{after}
% glue items.  The parbox is top-aligned, so the label text and the item
% run downwards together.  I put the result in box~0, and remove the depth,
% so as not to make the top line of the item text look really strange.
%
% All this leaves a little problem, though: if the item text isn't very long,
% the label might go further down the page than the main item, and possibly
% collide with the label below.  I must confess that I'm not actually sure
% how to deal with this possibility, so I just hope it doesn't happen.
%
% By the way, I don't have moral objections to |\parbox|.
%
%    \begin{macrocode}
\def\multilinelabel#1{%
  \setbox\z@\hbox{%
    \parbox[t]\labelwidth{\raggedright\hskip\z@skip#1}%
  }%
  \dp\z@\z@%
  \box\z@%
  \hfil%
}
%    \end{macrocode}
%
% \end{macro}
%
% \begin{macro}{\pushlabel}
%
% Now we implement the old style behaviour -- if the label is too wide, we
% just push the first line of the item further over to the right.  This
% is really very easy indeed -- we just stick some |\hfil| space on the
% right hand side (to left align if the label comes up too short).  The
% `push' behaviour is handled automatically by \LaTeX's item handling.
%
%    \begin{macrocode}
\def\pushlabel#1{{#1}\hfil}
%    \end{macrocode}
%
% \end{macro}
%
% \subsubsection{The main environment}
%
% \begin{macro}{\desclabelstyle}
%
% This is a declaration intended to be used only in the argument to the
% \env{basedescript} environment.  It sets the label style for the list.
% All we do is take the argument and assign it to a magic control sequence
% which \env{basedescript} will understand later.
%
%    \begin{macrocode}
\def\desclabelstyle#1{\def\desc@labelstyle{#1}}
%    \end{macrocode}
%
% \end{macro}
%
% \begin{macro}{\desclabelwidth}
%
% We set the label width and various other bits of information which will
% make all the bits of the description line up beautifully.  We set
% |\labelwidth| to the value we're given (using |\setlength|, so that
% people can use the \package{calc} package if they so wish), and make
% the |\leftmargin| equal $|\labelwidth|+|\labelsep|$.
%
%    \begin{macrocode}
\def\desclabelwidth#1{%
  \setlength\labelwidth{#1}%
  \leftmargin\labelwidth%
  \advance\leftmargin\labelsep%
}
%    \end{macrocode}
%
% \end{macro}
%
% \begin{environment}{basedescript}
%
% This is the new description environment.  It does almost everything you
% could want from a description environment, I think.  The argument is a
% collection of declarations to be performed while setting up the list.
%
% This environment isn't really intended to be used by users -- it's here
% so that you can define other description environments in terms of it,
%
% The environment is defined in two bits -- the `start' bit here simply
% starts the list and inserts the user declarations in an appropriate
% point, although sensible details will be inerted if the argument was
% empty.
%
%    \begin{macrocode}
\def\basedescript#1{%
%    \end{macrocode}
%
% We must start the list.  If the |\item| command's optional argument is
% missing, we should just leave a blank space, I think.
%
%    \begin{macrocode}
  \list{}{%
%    \end{macrocode}
%
% So far, so good.  Now put in some default declarations.  I'll use a
% separate macro for this, so that the global appearance of lists can be
% configured.
%
%    \begin{macrocode}
    \defaultdesc%
%    \end{macrocode}
%
% Now we do the user's declarations.
%
%    \begin{macrocode}
    #1%
%    \end{macrocode}
%
% Now set up the other parts of the list.  We set |\itemindent| so that the
% label is up against the current left margin.  (The standard version
% actually leaves the label hanging to the left of the margin by a
% distance of |\labelsep| for a reason I can't quite comprehend -- there's
% an |\hspace{\labelsep}| in the standard |\makelabel| to compensate for
% this.  Strange\dots)
%
% To make the label start in the right place, the text of the item must
% start a distance of $|\labelwidth|+|\labelsep|$ from the (pre-list) left
% hand margin; this means that we must set |\itemindent| to be
% $|\labelwidth|+|\labelsep|-|\leftmargin|$.  Time for some \TeX\ arithmetic.
%
%    \begin{macrocode}
    \itemindent\labelwidth%
    \advance\itemindent\labelsep%
    \advance\itemindent-\leftmargin%
%    \end{macrocode}
%
% Now we must set up the label typesetting.  We'll take the |\makelabel|
% provided by the user, remember it, and then redefine |\makelabel| in
% terms of the |\desclabelstyle| and the saved |\makelabel|.
%
%    \begin{macrocode}
    \let\desc@makelabel\makelabel%
    \def\makelabel##1{\desc@labelstyle{\desc@makelabel{##1}}}%
%    \end{macrocode}
%
% I can't think of anything else which needs doing, so I'll call it a day
% there.
%
%    \begin{macrocode}
  }%
}
%    \end{macrocode}
%
% Now we define the `end-bit' of the environment.  Since all we need to do
% is to close the list, we can be ever-so slightly clever and use |\let|.
%
%    \begin{macrocode}
\let\endbasedescript\endlist
%    \end{macrocode}
%
% Note that with these definitions, the standard \env{description}
% environment can be emulated by saying simply:
% \begin{listing}
%\renewenvironment{description}{%
%  \begin{basedescript}{}%
%}{%
%  \end{basedescript}
%}
% \end{listing}
%
% \end{environment}
%
% \begin{macro}{\defaultdesc}
%
% Now to set up the standard description appearance.  In the absence
% of any other declarations, the label will `push' the text out the way if
% the text is too long.  The standard |\labelsep| and |\leftmargin| are not
% our problem.  We typeset the label text in bold by default. Also,
% |\labelwidth| is cleared to 0\,pt, because this is what \LaTeX's usual
% \env{description} does.
%
%    \begin{macrocode}
\providecommand\defaultdesc{%
  \desclabelstyle\pushlabel%
  \def\makelabel##1{\bfseries##1}%
  \labelwidth\z@%
}
%    \end{macrocode}
%
% \end{macro}
%
% \subsubsection{An example}
%
% \begin{environment}{note}
%
% The \env{note} environment is a simple application of the general
% description list shown above.  It typesets the label (by default, the
% text `\textbf{note}') at the left margin, and the note text indented by
% the width of the label.
%
% The code is simple -- we take the environment's argument (which may have
% been omitted), store it in a box (using |\sbox| again, to handle colour
% changes correctly), set the label width from the width of the box, and
% then create a single item containing the label text.  The text of the
% environment then appears in exactly the desired place.
%
% I've not used |\newcommand| here, for the following reasons:
% \begin{itemize}
%
% \item I don't like it much, to be honest.
%
% \item Until very recently, |\newcommand| only allowed you to define
%       `long' commands, where new paragraphs were allowed to be started
%       in command arguments; this removes a useful check which traps
%       common errors like missing out `|}|' characters.  I'd prefer to
%       be compatible with older \LaTeX s than to use the new |\newcommand|
%       which provides a $*$-form to work around this restriction.
%
% \end{itemize}
%
%    \begin{macrocode}
\def\note{\@ifnextchar[\note@i{\note@i[Note:]}}
\def\note@i[#1]{%
  \basedescript{%
    \sbox\z@{\makelabel{#1}}%
    \desclabelwidth{\wd\z@}%
  }%
  \item[\box\z@]%
}
\let\endnote\endbasedescript
%    \end{macrocode}
%
% \end{environment}
%
%
% \subsection{Compacted environments}
%
% Normal lists tend to have rather too much space between items if all or
% most of the item texts are one line or less each.  We therefore define
% a macro |\makecompactlist| whuch creates `compacted' versions of existing
% environments.
%
% \begin{macro}{\makecompactlist}
%
% We're given two arguments: the name of the new environment to create, and
% the name of the existing list environment to create.
%
% The first thing to do is to ensure that the environment we're creating is
% actually valid (i.e., it doesn't exist already, and it has a sensible
% name).  We can do this with the internal \LaTeX\ macro |\@ifdefinable|.
%
%    \begin{macrocode}
\def\makecompactlist#1#2{%
  \expandafter\@ifdefinable\csname#1\endcsname%
    {\makecompactlist@i{#1}{#2}}%
}
%    \end{macrocode}
%
% We also ought to ensure that the other environment already exists.  This
% isn't too tricky.  We'll steal \LaTeX's error and message for this.
%
%    \begin{macrocode}
\def\makecompactlist@i#1#2{%
  \@ifundefined{#2}{\me@err{Environment `#2' not defined}\@ehc}{}%
%    \end{macrocode}
%
% The main work for starting a compact list is done elsewhere.
%
%    \begin{macrocode}
  \@namedef{#1}{\@compact@list{#2}}%
%    \end{macrocode}
%
% Now to define the end of the environment; this isn't terribly difficult.
%
%    \begin{macrocode}
  \expandafter\let\csname end#1\expandafter\endcsname%
                  \csname end#2\endcsname%
%    \end{macrocode}
%
% That's a compacted environment created.  Easy, no?
%
%    \begin{macrocode}
}
%    \end{macrocode}
%
% The general case macro has to try slurping some arguments, calling the
% underlying environment, and removing vertical space.
%
%    \begin{macrocode}
\def\@compact@list#1{\@testopt{\@compact@list@i{#1}}{}}
\def\@compact@list@i#1[#2]{%
  \@nameuse{#1}#2%
  \parskip\z@%
  \itemsep\z@%
}%
%    \end{macrocode}
%
% \end{macro}
%
% \begin{environment}{itemize*}
% \begin{environment}{enumerate*}
% \begin{environment}{description*}
%
% Let's build some compacted environments now.  These are easy now that
% we've done all the work above.
%
%    \begin{macrocode}
\makecompactlist{itemize*}{itemize}
\makecompactlist{enumerate*}{enumerate}
\makecompactlist{description*}{description}
%    \end{macrocode}
%
% \end{environment}
% \end{environment}
% \end{environment}
%
%
% \subsection{Suspending and resuming lists}
%
% This is nowhere near perfect; it relies a lot on the goodwill of the user,
% although it seems to work fairly well.
%
% \begin{macro}{\suspend}
%
% The only thing that needs saving here is the list counter, whose name
% is stored in |\@listctr|.  When I get a request to save the counter, I'll
% build a macro which will restore it when the environment is restored later.
%
% The first thing to do is to handle the optional argument.  |\@dblarg| will
% sort this out, giving me a copy of the mandatory argument if there's no
% optional one provided.
%
%    \begin{macrocode}
\def\suspend{\@dblarg\suspend@i}
%    \end{macrocode}
%
% That's all we need to do here.
%
%    \begin{macrocode}
\def\suspend@i[#1]#2{%
%    \end{macrocode}
%
% Now I have a little problem; when I |\end| the environment, it will close
% off the grouping level, and the counter value will be forgotten.  This is
% bad.  I'll store all my definitions into a macro, and build the |\end|
% command into it; that way, everything will be expanded correctly.  This
% requires the use of |\edef|, which means I must be a little careful.
%
%    \begin{macrocode}
  \edef\@tempa{%
%    \end{macrocode}
%
% The first thing to do is to end the environment.  I don't want |\end|
% expanded yet, so I'll use |\noexpand|.
%
%    \begin{macrocode}
    \noexpand\end{#2}%
%    \end{macrocode}
%
% Now I must define the `resume' macro.  I'll use |\csname| to build the
% named identifier into the name, so it won't go wrong (maybe).  There's
% a little fun here to make the control sequence name but not expand it
% here.
%
%    \begin{macrocode}
    \def\expandafter\noexpand\csname resume.#1\endcsname{%
%    \end{macrocode}
%
% The counter name is hidden inside |\@listctr|, so the actual counter is
% called `|\csname c@\@listctr\endcsname|'.  I'll use |\the| to read its
% current value, and assign it to the counter when the macro is used later.
%
%    \begin{macrocode}
      \csname c@\@listctr\endcsname\the\csname c@\@listctr\endcsname%
%    \end{macrocode}
%
% That's all we need to do there.  Now close the macros and run them.
%
%    \begin{macrocode}
    }%
  }%
  \@tempa%
}
%    \end{macrocode}
%
% \end{macro}
%
% \begin{macro}{\resume}
%
% Resuming environments is much easier.  Since I use |\csname| to build the
% name, nothing happens if you try to resume environments which weren't
% suspended.  I'll trap this and raise an error.  Provide an optional
% argument for collecting arguments to the target list.
%
%    \begin{macrocode}
\def\resume{\@dblarg\resume@i}
\def\resume@i[#1]#2{\@testopt{\resume@ii{#1}{#2}}{}}
\def\resume@ii#1#2[#3]{%
  \begin{#2}#3%
  \@ifundefined{resume.#1}{\ml@err@resume}{\@nameuse{resume.#1}}%
}
%    \end{macrocode}
%
% \end{macro}
%
% That's all there is.
%
%    \begin{macrocode}
%</package>
%    \end{macrocode}
%
% \hfill Mark Wooding, \today
%
% \Finale
%
\endinput
