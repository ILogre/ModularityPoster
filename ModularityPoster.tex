\documentclass{acm_proc_article-sp}
\usepackage{seb}
\usepackage{fancyref}

\begin{document}


\title{Composition Challenges for Sensor Data Visualization}
% \subtitle{}

\numberofauthors{3} \author{
  % 1st. author
  \alignauthor Ivan Logre\\
  \affaddr{Univ. Nice Sophia Antipolis}\\
  \affaddr{CNRS, I3S, UMR 7271}\\
  \affaddr{06900 Sophia Antipolis}\\
  \email{logre@i3s.unice.fr}
  % 2nd. author
  \alignauthor S{\'e}bastien Mosser\\
  \affaddr{Univ. Nice Sophia Antipolis}\\
  \affaddr{CNRS, I3S, UMR 7271}\\
  \affaddr{06900 Sophia Antipolis}\\
  \email{mosser@i3s.unice.fr}
  % 3rd. author
  \alignauthor Michel Riveill\\
  \affaddr{Univ. Nice Sophia Antipolis}\\
  \affaddr{CNRS, I3S, UMR 7271}\\
  \affaddr{06900 Sophia Antipolis}\\
  \email{riveill@i3s.unice.fr} }

\date{06 January 2015}

\maketitle
\begin{abstract}
  %% Problem
  Connected objects and other monitoring systems continuously produce
  data about their environment. Dashboards are then designed to
  aggregate and present data to end-users.
  %% Contribution
  Technologies used to design and implement visualization dashboards
  are babbling from a software engineering point of view. This paper
  highlights how this domain could benefit from leveraging separation
  of concerns and software composition paradigms.
  %% Results
  % It describes how the design of dashboard visualizing data
  % collected from sensors triggers multiple challenges about software
  % composition through separation of concern and modularity in order to
  % insure properties on the design process.
\end{abstract}

% A category with the (minimum) three required fields
\category{D.2.8}{Software Engineering}{Data visualization, Software
  composition}[]

\keywords{visualization, sensor, data, composition}

\section{Introduction}
The increase of data producing objects leads to a growing need to
interpret those data in order to extract knowledge. When there is a
human at the end of the production chain, the result must allow her to
exploit it, so being adapted to his capabilities. Such visualization
are used to assist one to make a decision according to a projection of
facts, as are sensed data. Visualization is a way to allow a human to
induce information from raw data through dashboards. The first chapter
defines our context of work through a short definition of this field,
a clarification about the aim and the efforts to achieve it and
several exiting way to implement a visualization solution. The second
chapter raises the challenges relatively to software engineering,
detailing properties to observe while contributing to this particular
field. Then, the third chapter briefly offers some perspectives to
tackle those challenges.

\section{Visualization Dashboards}
% une colonne max.
\mypar{Process overview}
% Qui sont les roles intervenants?
In order to design and implement such visualization dashboard, there
is a need to chose visualizations to apply to refined datasets in
order to achieve some identified final goal. This task is to be
performed by three roles:
\begin{enumerate}
\item a Requirement Engineer (RE),
\item a Data Manager (DM),
\item a Dashboard Designer (DD).
\end{enumerate}
% Quels sont leurs domaines de competences?
Each role is responsible for several tasks: (i) the RE define what
will be the purpose of the dashboard and audit the resulting dashboard
according to the initial motivation, (ii) the DM select data to be
visualize and treat it in order to provide the needed well formatted
datasets, (iii) the DD chose visualization for each group of data and
arrange them spatially into a dashboard.
% Pourquoi ces competences sont souvent des acteurs differents et pas
% juste des roles?
Even if those roles collaborates, e.g. the choice of visualization the
DD has to make depends on the purpose exposed by the RE, they relates
on distinct domains of expertise and formation, each one bringing is
own challenges, therefore each role is usually impersonated by a
dedicated stakeholder.  \mypar{Designing dashboard}
% \begin{enumerate}
%   Quels sont les outils a dispositions (UML, CTT)
One could use existing solutions to implement each of the mentioned
tasks, for example using SQL would be a suitable choice to store and
refine sensor data, the Interaction Flow Modeling Language
(IFML)~\cite{} to model the wanted organization of the dashboard and
temporal logic or Concurrent Tasks Tree (CTT)~\cite{} to define the
aimed product and succession of action to be performed with it.
% Quels sont les problemes quand on les utilise?
Those tools were not designed for the visualization field and reveal a
lack of expressivity being too generic, as CTT that offers no solution
about characterization of visualization need.
% Est-ce qu'on sort certains experts de leur domaine de competences
% avec ces outils?
Moreover, there is a lack of interoperability between those tools due
to the distinct field they come from. This results in a difficulty for
the stakeholders to dialogue and converge toward a solution when a
compromise is needed.
% \end{enumerate}

\mypar{Implementing dashboard}
% \begin{enumerate}
% \item Quels sont les outils a dispositions (Web, bibliotheques)
To implement a given dashboard, one can use online visualization widget
libraries, either professional solution as HighChart and AmChart or
community-based example as D3.JS, and add HTML5 code to structure the
result.
% \item Quels sont les problemes quand on les utilise?
About the last one, due to the huge amount of available widgets
(e.g. D3.js offers 235 widgets on January 2015), the effort needed to
choose a suitable visualization is not suitable to a dashboard design
process~\cite{ecmfa}.
% \item Est-ce qu'on sort certains experts de leur domaine de
%   competences avec ces outils?
% \end{enumerate}


\section{Challenges}

The previous section described how people support the design and
implementation of dashboards using both state of the art and state of
practice methods. We highlight in this section four challenges
triggered by such approach, and discuss how separation of concerns and
software composition could be helpful to support this process.

%% Pour chaque paragraphe, se raccrocher au contexte de la section
%% d'avant (je ne peux pas le faire tant que c'est aps ecrit),
%% identifier le probleme du point de vue ou de la separation ou de la
%% composition (ou des deux), et discuter des benefics envisages par
%% une approche de ce genre.

\mypar{Evolution capabilities}
Data visualization is a growing, fast evolving field, as well as
requirement engineering is an active research field offering new way
to capture needs.  For those reasons, their respective DSL will have
to evolve in time, the composition should therefore take this fact
into consideration to ease the evolutivity of the whole composed
dashboard design framework.

\mypar{Isolation}
Each domain expert (i.e. user of a solution specific to a domain)
should be able to work in isolation from the other domains. This means
that one can always contribute in her own domain, to improve her
contribution, and being able to check that this evolution is
consistent in this domain, even if the other domains are in an
unstable or incoherent state.

\mypar{Low coupling}
The number of hypothesis made in a domain implementation to the others
should be as small as possible. In addition to be a good software
development practice, this criteria eases the replacement of solution
by an other one, e.g. of the data management by a visualization
resource DSL.
%%% to complete

\mypar{Integrity}

\section{Conclusions}

%% Un petit paragraphe de conclusions/

%% Ouvrir sur la compo de DSL
\mypar{Actionable insights: DSL composition}
One interesting way to tackle those challenges would be to design a
DSL for each of the three domains mentioned in \ref{subsec:How to
  implement it} and then to compose those partial results in a overall
data visualization solution.
The state of the art reveals two main ways to manage this composition :\\
\iitem{i} Fusion/Merge, i.e. the operation to produce one bigger meta
model from several meta models by identification of a pivot and
merging from it, and
then refine the associated concrete syntaxes to produce a global one. (insert ref)\\
\iitem{ii} Aggregation, i.e. make enough assumptions about the meta models to be able to link them through the transformation of several meta models by adding, deleting or editing specific model elements, essentially to align two concepts from different domains or to reference an external concept in order to delegate part of the responsibilities. (insert ref)\\
The authors short-term perspective is to work on the integration of
the DSLs as software services in order to be able to compose languages
while insuring the same properties as service-oriented architecture
field.



% The following two commands are all you need in the initial runs of
% your .tex file to produce the bibliography for the citations in your
% paper.
\bibliographystyle{abbrv}
\bibliography{sigproc} % sigproc.bib is the name of the Bibliography in this case
% You must have a proper ".bib" file and remember to run: latex bibtex
% latex latex to resolve all references
%
% ACM needs 'a single self-contained file'!
\end{document}
