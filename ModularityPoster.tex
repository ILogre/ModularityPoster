\documentclass{acm_proc_article-sp}

\begin{document}

\title{Composition challenges for sensor data visualization}
%\subtitle{}

\numberofauthors{4}
\author{
% 1st. author
\alignauthor Ivan Logre\\
       \affaddr{Univ. Nice Sophia Antipolis}\\
       \affaddr{CNRS, I3S, UMR 7271}\\
       \affaddr{06900 Sophia Antipolis}\\
       \email{logre@i3s.unice.fr}
% 2nd. author
\alignauthor S{\'e}bastien Mosser\\
       \affaddr{Univ. Nice Sophia Antipolis}\\
       \affaddr{CNRS, I3S, UMR 7271}\\
       \affaddr{06900 Sophia Antipolis}\\
       \email{mosser@i3s.unice.fr}
\and
% 3rd. author
\alignauthor Philippe Collet\\
       \affaddr{Univ. Nice Sophia Antipolis}\\
       \affaddr{CNRS, I3S, UMR 7271}\\
       \affaddr{06900 Sophia Antipolis}\\
       \email{collet@i3s.unice.fr}
% 4th. author
\alignauthor Michel Riveill\\
       \affaddr{Univ. Nice Sophia Antipolis}\\
       \affaddr{CNRS, I3S, UMR 7271}\\
       \affaddr{06900 Sophia Antipolis}\\
       \email{riveill@i3s.unice.fr}
}

\date{06 January 2015}

\maketitle
\begin{abstract}
This paper provides a overview of the sensor data visualization requirements in
terms of software engineering. The first chapter defines our context of work
through a short definition of this field, a clarification about the aim and the
efforts to achieve it and several exiting way to implement a visualization solution.
The second chapter raises the challenges relatively to software engineering,
detailing properties to observe while contributing to this particular field.
Then, the third chapter briefly offers some perspectives to tackle those challenges.
\end{abstract}

% A category with the (minimum) three required fields
\category{D.2.8}{Software Engineering}{Data visualization, Software composition}[]

\terms{???}

\keywords{visualization, sensor, data, composition}

\section{Introduction}
Introduction

\section{Sensor data visualization}
The increase of data producing objects leads to a growing need to interpret those
data in order to extract knowledge. When there is a human at the end of the
production chain, the result must allow her to exploit it, so being adapted to his
capabilities. Such visualization are used to assist one to make a decision according
to a projection of facts, as are sensed data.

\subsection{What is sensor data visualization ?}
% Data visualization through dashboards (def)
Visualization is the most common way to allow a human to induce information from raw data
through dashboards.
Stephen Few define a dashboard as "a visual display of the most important
information needed to achieve one or more objectives; consolidated and arranged
on a single screen so the information can be monitored at a glance"\cite{few:dashboard} \\
% What kind of visualizations ?
According to this definition, the visualizations used for this purpose should
be a graphical representation of the data to treat. Apart from the well-known
generic charts (e.g. line chart, bar chart, pie chart, ...), a huge amount
of specialized visualizations is available for specific data and/or task.
% Static vs dynamic
Such graphical widgets may present a static representation of the data, as an
assessment of a situation, or may offer one to interact with it with an aim of
exploration, to search a more relevant data for each use, or to offer both an
overview and a way to reduce the scope of the visualization.
% Nb of dimensions
Besides the ornamental criteria which only consist on a effect to change the aspect of the widget without impacting it capabilities (eg. 3D pie chart), the number of dimensions displayed play a part in the final use of the visualizations. Indeed, there may be one dimension (e.g. a simple display of your current speed on your car dashboard), two dimensions (as the generic charts above), or more (e.g. a country map with a color gradient to weight the population). 
% For what kind of data ?


% For what purposes ?

\subsection{How to achieve it ?}
Abstract description of the task to perform
Lots of methodologies, but need to chose visualizations to apply to refined
data sets in order to achieve some final goal
		-> several challenges
		define what is the information the dashboard should help understanding
		prepare data to be visualize
		chose visualization for each group of data
		fill the visualization with the well formatted data set
		arrange the visualisations into a (multiple) dashboard
		audit the dashboard according to the initial motivation

\subsection{How to implement it}
		Those challenges refer to three different domains :
			Requirement Engineering
			Data treatment
			Dashboard design
		Existing : SQL...
		One other way to tackle it is to design one DSL per domain
This way, one, when in charge of a given domain, may deal with a language
specifically designed for his task, without being polluted by others
vocabulary or domain constraints.
%/!\ � traiter par domaine ou par "d�marche"

\section{Properties to satisfy and associated challenges}
However, the design of such visualization tool is not trivial, leading to several challenges.

\subsection{Evolutivity}
Data visualization is a growing, fast evolving field, as well as requirement
engineering is an active research field offering new way to capture needs.
For those reasons, their respective DSL will have to evolve in time, the
composition should therefore take this fact into consideration to ease the
evolutivity of the whole composed dashboard design framework.

\subsection{Isolation}
\subsection{Low coupling}
\subsection{Integrity}

\section{Actionable insights}
\subsection{DSL composition}
		Aggregation
		Fusion/Merge
		Integration



%\end{document}  % This is where a 'short' article might terminate


%
% The following two commands are all you need in the
% initial runs of your .tex file to
% produce the bibliography for the citations in your paper.
\bibliographystyle{abbrv}
\bibliography{sigproc}  % sigproc.bib is the name of the Bibliography in this case
% You must have a proper ".bib" file
%  and remember to run:
% latex bibtex latex latex
% to resolve all references
%
% ACM needs 'a single self-contained file'!
\end{document}
