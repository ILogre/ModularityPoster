\documentclass{acm_proc_article-sp}
\usepackage{seb}
\usepackage{fancyref}

\begin{document}


\title{Composition Challenges for Sensor Data Visualization}
% \subtitle{}

\numberofauthors{3} \author{
  % 1st. author
  \alignauthor Ivan Logre\\
  \affaddr{Univ. Nice Sophia Antipolis}\\
  \affaddr{CNRS, I3S, UMR 7271}\\
  \affaddr{06900 Sophia Antipolis}\\
  \email{logre@i3s.unice.fr}
  % 2nd. author
  \alignauthor S{\'e}bastien Mosser\\
  \affaddr{Univ. Nice Sophia Antipolis}\\
  \affaddr{CNRS, I3S, UMR 7271}\\
  \affaddr{06900 Sophia Antipolis}\\
  \email{mosser@i3s.unice.fr}
  % 3rd. author
  \alignauthor Michel Riveill\\
  \affaddr{Univ. Nice Sophia Antipolis}\\
  \affaddr{CNRS, I3S, UMR 7271}\\
  \affaddr{06900 Sophia Antipolis}\\
  \email{riveill@i3s.unice.fr} }

\date{06 January 2015}

\maketitle
\begin{abstract}
  %% Problem
  Connected objects and other monitoring systems continuously produce
  data about their environment. Dashboards are then designed to
  aggregate and present data to end-users.
  %% Contribution
  Technologies used to design and implement visualization dashboards
  are babbling from a software engineering point of view. This paper
  highlights how this domain could benefit from leveraging separation
  of concerns and software composition paradigms.
  %% Results
  % It describes how the design of dashboard visualizing data
  % collected from sensors triggers multiple challenges about software
  % composition through separation of concern and modularity in order to
  % insure properties on the design process.
\end{abstract}

% A category with the (minimum) three required fields
\category{D.2.8}{Software Engineering}{Data visualization, Software
  composition}[]

\keywords{visualization, sensor, data, composition}

\section{Introduction}

The Internet of Things (IoT) relies on physical objects interconnected
between each others, creating a mesh of devices producing
information. In this context, sensors are surrounding our environment
(\eg cars, buildings, smartphones) and continuously collect data about
our living environment. In order to add value to these raw data sets,
visualization dashboards are designed to support end-user decision
making process. Unfortunately, the tools available to design and
implement such dashboards are holistic, and does not take into account
the inherent modularity of this domain. This paper does not aim to
describe a solution, but instead focuses on the challenges triggered
by the design of visualization dashboard, and align them with the
modular paradigms such as separation of concerns and software
composition.

% The increase of data producing objects leads to a growing need to
% interpret those data in order to extract knowledge. When there is a
% human at the end of the production chain, the result must allow her to
% exploit it, so being adapted to his capabilities. Such visualization
% are used to assist one to make a decision according to a projection of
% facts, as are sensed data. Visualization is a way to allow a human to
% induce information from raw data through dashboards. The first chapter
% defines our context of work through a short definition of this field,
% a clarification about the aim and the efforts to achieve it and
% several exiting way to implement a visualization solution. The second
% chapter raises the challenges relatively to software engineering,
% detailing properties to observe while contributing to this particular
% field. Then, the third chapter briefly offers some perspectives to
% tackle those challenges.

\section{Visualization Dashboards}
% une colonne max.
\mypar{Process overview}
% Qui sont les roles intervenants?
In order to design and implement such visualization dashboard, there
is a need to chose visualizations to apply to refined datasets in
order to achieve some identified final goal. This task is to be
performed by three roles:
\begin{enumerate}
\item a Requirement Engineer (RE);
\item a Data Manager (DM);
\item a Dashboard Designer (DD).
\end{enumerate}
% Quels sont leurs domaines de competences?
Each role is responsible for several tasks: (i) the RE define what
will be the purpose of the dashboard and audit the resulting dashboard
according to the initial motivation, (ii) the DM select data to be
visualize and treat it in order to provide the needed well formatted
datasets, (iii) the DD chose visualization for each group of data and
arrange them spatially into a dashboard.
% Pourquoi ces competences sont souvent des acteurs differents et pas
% juste des roles?
Even if those roles collaborates, \eg the choice of visualization the
DD has to make depends on the purpose exposed by the RE, they relates
on distinct domains of expertise and formation, each one bringing is
own challenges, therefore each role is usually impersonated by a
dedicated stakeholder.  \mypar{Designing dashboard}
% \begin{enumerate}
%   Quels sont les outils a dispositions (UML, CTT)
One could use existing solutions to implement each of the mentioned
tasks, for example using SQL would be a suitable choice to store and
refine sensor data, the Interaction Flow Modelling Language
(IFML)~\cite{ifml} to model the wanted organization of the dashboard and
temporal logic or Concurrent Tasks Tree (CTT)~\cite{ctt} to define the
aimed product and succession of action to be performed with it.
% Quels sont les problemes quand on les utilise?
Those tools were not designed for the visualization field and reveal a
lack of expressiveness being too generic, as CTT that offers no solution
about characterization of visualization need.
% Est-ce qu'on sort certains experts de leur domaine de competences
% avec ces outils?
Moreover, there is a lack of interoperability between those tools due
to the distinct field they come from. This results in a difficulty for
the stakeholders to dialogue and converge toward a solution when a
compromise is needed.
% \end{enumerate}

\mypar{Implementing dashboard}
% Quels sont les outils a dispositions (Web, bibliotheques)
To implement a given dashboard, one can use online visualization widget
libraries, either professional solution as HighChart and AmChart or
community-based example as D3.JS, and add HTML5 code to structure the
result.
% Quels sont les problemes quand on les utilise?
However, those widgets does not allows integrability with a lot of data format,
since the development effort is put on the interaction more instead of the interoperability.
Also, the huge amount of available widgets, \eg D3.js offers 235 widgets on
January 2015, increase the difficulty to choose a suitable visualization,
% Est-ce qu'on sort certains experts de leur domaine de competences avec ces outils?
due to the lack of effort in categorization of those new visualization
capabilities\cite{ecmfa}.
Those last two points strengthen the difficulty to cooperate with other domains
because of the gap between the conceptual role of the RE and the implementation
role of the DD, and because of the incompatible constraints impose by libraries
on data format which reduce re-usability.

\section{Challenges}

The previous section described how people support the design and
implementation of dashboards using both state of the art and state of
practice methods. We highlight in this section the challenges of isolation
triggered by such approach, detailing it through two axes, namely evolution
capacity and integrity, and discuss how separation of concerns and
software composition could be helpful to support this process.

%% Pour chaque paragraphe, se raccrocher au contexte de la section
%% d'avant (je ne peux pas le faire tant que c'est aps ecrit),
%% identifier le probleme du point de vue ou de la separation ou de la
%% composition (ou des deux), et discuter des benefics envisages par
%% une approche de ce genre.
\mypar{Isolation challenge}
Each stakeholder should be able to work in his domain of competence,
in isolation from the other domains. This main challenge implies that
each one may work without the irrelevant noise of others contribution,
focusing on concepts of his own domain, or, in case of a shared concept,
only on the facet of this concept relevant for the task of this role.
For example, as a dashboard designer, one may want to define a link
between the visualization being constructed and some data, but should
not be requested to be competent as a data manager while working on it.
Indeed, the relevant informations in a data polishing point of view should
be handled by a specialist of this domain, but not only this specialist
would need to reference a specific dataset.
% quels sont les avantages a les maintenir séparés ^

\mypar{Evolution capabilities}
Data visualization is a growing, fast evolving field
(6 440 000 Google results for \emph{"data visualization"} in 01/15,
D3.js offered 133 widgets in 04/14 and 235 in 01/15),
as well as requirement engineering (RE) is an active research field
offering new way to capture needs (more than 730 papers published
since 2014 and referenced by Google Scholar contains
\emph{"requirement engineering"}, and 640 \emph{"goal model"})
and data scalability is still an opened scientific lock.
For those reasons, their respective tool or domain
specific languages (DSL) used will have to evolve in time, the composition
should therefore take this fact into consideration to ease the evolution
capability of the whole composed dashboard design framework.
Each of this domains bring unique competences and expertise useful to
the data visualization design field. Keeping the separation of concern
allows each stakeholder to perform her dedicated task with the state of the art
capabilities of her domain.
Nonetheless, this separation of role specific solutions implies a composition
of those partial results which could bridge the gap between the domains and
manage the high versatility of those research fields.

\mypar{Integrity}
In addition, the separation of concern has to allow one to contribute
in her own domain, while being able to check if this evolution is
consistent locally, meaning inside this domain, even if the other domains
are in an unstable or incoherent state or if it is not possible to check
the global consistency at the moment. For example, one as a data manager
should be allowed to edit a resource measurement unit to optimize the data
transfer and validate his contribution in the data point of view,
even if it may have break the choice of visualizations used in the dashboard
design domain.
In order to reach a proper separation of concern, each domain solution 
has then to be usable independently from the whole composed approach.
This point raises a challenge on the composition of this partial solutions,
introducing the notion of local and global consistency to handle.

%% Ouvrir sur la compo de DSL
\mypar{Actionable insights: DSL composition}
One interesting way to tackle those challenges would be to design a
DSL for each of the three domains mentioned and then to compose those
partial results in a overall data visualization solution.
The state of the art reveals two main ways to manage this composition :\\
\iitem{i} Fusion/Merge, i.e. the operation to produce one bigger meta
model from several meta models by identification of a pivot and
merging from it, and
then refine the associated concrete syntaxes to produce a global one. \cite{kienzle13}\\
\iitem{ii} Aggregation, i.e. make enough assumptions about the meta models to be able to link them through the transformation of several meta models by adding, deleting or editing specific model elements, essentially to align two concepts from different domains or to reference an external concept in order to delegate part of the responsibilities. \cite{blouin14}\\
The authors short-term perspective is to work on the integration of
the DSLs as software services in order to be able to compose languages
while insuring the same properties as service-oriented architecture
field.

\section{Conclusions \& Perspectives}
In this paper, we described the domain of visualization dashboard
design. This domain crosscuts several research fields, from
human-computer interactions to big data. We highlighted four
challenges in this domain where a software engineering approach based
on modularity concepts could support it. However, all the challenges
triggered by this domain are not yet solved from a separation of
concerns point of view. In our upcoming works, we plan to focus on
defining a formal way to support exchanges between the different roles
involved in the domain, emphasizing integrity and isolation
properties.
%% Un petit paragraphe de conclusions/



% The following two commands are all you need in the initial runs of
% your .tex file to produce the bibliography for the citations in your
% paper.
\bibliographystyle{abbrv}
\bibliography{sigproc} % sigproc.bib is the name of the Bibliography in this case
% You must have a proper ".bib" file and remember to run: latex bibtex
% latex latex to resolve all references
%
% ACM needs 'a single self-contained file'!
\end{document}
